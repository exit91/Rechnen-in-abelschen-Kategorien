\section{Faserprodukte (und Fasersummen)}

Ein wenig Notation: Ich bezeichne für Pfeile $f : X\to Z, g: Y\to Z$ die Projektionen des Faserprodukts als $\fib_1(f,g) : X\times_Z Y \to X$ und $\fib_2(f,g) : X\times_Z Y \to Y$. Weiter definiere $\fib(f,g) := f.\fib_1(f,g) = g.\fib_2(f,g)$.
Es kommutieren also notwendigerweise die Argumente $\fib(f,g) = \fib(g,f)$.
Für Fasersummen ersetze ich $\fib$ durch $\cofib$.
Um etwas mit dieser Notation vertraut zu werden verwende ich sie gleich in der folgenden Definition.

\begin{defn}[Universelle Eigenschaft des Faserprodukts]
Seien $f:X\to Z, g:Y\to Z$ Pfeile, dann heißt das Tripel $(X\times_Z Y, \fib_1(f,g), \fib_2(f,g))$ bestehend aus einem Objekt und zwei Projektionen \emph{Faserprodukt}, falls:
\begin{enumerate}
\item $f.\fib_1(f,g) = g.\fib_2(f,g)$
\item Für jedes weitere Paar von Morphismen $\theta_1 : M \to X, \theta_2 : M \to X$ erhalten wir einen eindeutigen Pfeil $\mu : M \to X\times_Z Y$, sodass $\theta_i = \fib_i(f,g) . \mu$ für $i=1,2$.
\end{enumerate}
Dies definiert ein Faserprodukt eindeutig bis auf eindeutigen Iso.
\end{defn}

\begin{prop} Sei $\mathcal C$ eine Kategorie, $f : X \to Z, g : Y\to Z$ Morphismen sowie $X\times_Z Y$ deren Faserprodukt. Ist $g: Y\to Z$ ein Mono oder Iso, so ist auch die Projektion $\fib_1(f,g): X\times_Z Y \to X$ ein Mono bzw. ein Iso.
\end{prop}

\begin{prop}
Ein Spezialfall: $\fib_1(f,\id_Y) = f$ und $\fib_2(f,\id_Y) = \id_X$ für $f : X\to Y$.
\end{prop}

\begin{korr}
Sei $\mathcal A$ eine abelsche Kategorie und
$X_1 \hookr X$ und $X_2 \hookr X$ Monos in $\mathcal A$. Dann ist der induzierte Pfeil $X_1 \times_X X_2 \to X$ auch Mono als Komposition $X_1\times_X X_2 \hookr X_2 \hookr X$ von Monos.
\end{korr}

\begin{prop} Sei $\mathcal A$ eine abelsche Kategorie, $f : X \to Z, g : Y\to Z$ Morphismen. Ist $g: Y\to Z$ ein Epi, so ist auch die Projektion $\fib_1(f,g): X\times_Z Y \to X$ ein Epi.
\end{prop}

\begin{satz}[Staffelung]
Seien $f: X\to Y,g : Y\to Z,h:W\to Z$ Pfeile und (bla, Existenzaussage damit alle folgenden Faserprodukte definiert sind...). Dann ist $X\times_{g.f,Z,h} W = X\times_{f,Y,\fib_1(g,h)} (Y\times_{g,Z,h} W)$ und es gelten
\begin{itemize}
\item $\fib_1(g.f,h) = \fib_1(f,\fib_1(g,h))$
\item $\fib_2(g.f,h) = \fib_2(g,h) . \fib_2(f,\fib_1(g,h))$
\end{itemize}
\end{satz}

\begin{prop}[Kommutativität]
Seien $f : X\to Z, g:Y\to Z$ Pfeile. Dann gilt
\[ X\times_Z Y = Y\times_Z X \]
mit $\fib_1(f,g) = \fib_2(g,f)$ und $\fib_2(f,g) = \fib_1(g,f)$.
An dieser Stelle ist es sinnvoll nochmal zu erwähnen, dass $\fib(f,g) = \fib(g,f)$.
\end{prop}

\begin{prop}[Assoziativität]
Seien $f : W \to Z, g: X \to Z, h : Y\to Z$ Pfeile und $\{ \alpha, \beta, \gamma \} = \{ f,g,h\}$ eine beliebige Permutation dieser Pfeile. Dann gelten
\[ \fib(\fib(f,g),\fib(f,h)) = \fib(\fib(\alpha,\beta),\fib(\alpha,\gamma)). \]
Genauer haben wir sogar
\[ \fib_1(\fib_1(f,g),\fib_1(f,h)) = \fib_1(\fib_1(g,f),\fib_1(g,h)) \]
und schließlich gilt auch noch
\[ \fib(\fib(f,g),h) = \fib(f,\fib(g,h)) \]
\end{prop}

\begin{korr}[Faserproduktwürfel]
Im folgenden werden wir etwas Vorstellungskraft brauchen: Mit $E_{000},E_{001},...,E_{111}$ werden wir Objekte bezeichnen, die auf den Ecken eines Einheitsquadrats $[0,1]^3$ sitzen, nämlich an den Punkten $(0,0,0),(0,0,1),...(1,1,1)$.
Gegeben seien drei Pfeile $f : E_{100} \to E_{110}, g : E_{111} \to E_{110}, h : E_{010} \to E_{110}$ gegeben. Die Behauptung ist nun, dass es möglich ist, jeder verbleibenden Kante genau einen Pfeil und jeder verbleibenden Ecke ein Objekt zuzuordnen, sodass auf jeder Seite des Würfels ein Faserproduktdiagramm steht.

%TODO
[Hier kommt eine nette Illustration hin]
\end{korr}

\begin{prop}
Für $f: X\to Z, g:Y\to Z, h:Z\to Z'$ existiert ein eindeutiger Pfeil $\kappa : X\times_{f,Z,g} Y \to X\times_{h.f,Z,h.g} Y$ mit $\fib_i(f,g) = \fib_i(h.f,h.g).\kappa$ für $i=1,2$. Ist auch noch $h$ ein Mono, dann ist $\kappa$ ein Iso.
\end{prop}

