\section{Faserprodukte (und Fasersummen)}

Ein wenig Notation: Ich bezeichne für Pfeile $f : X\to Z, g: Y\to Z$ die Projektionen des Faserprodukts als $\fib(f,g)_1 : X\times_Z Y \to X$ und $\fib(f,g)_2 : X\times_Z Y \to Y$. Um etwas mit dieser Notation vertraut zu werden verwende ich sie gleich in der folgenden Definition.

Für Fasersummen ersetze ich $\fib$ durch $\cofib$.

\begin{defn}[Universelle Eigenschaft des Faserprodukts]
Seien $f:X\to Z, g:Y\to Z$ Pfeile, dann heißt das Tripel $(X\times_Z Y, \fib(f,g)_1, \fib(f,g)_2)$ bestehend aus einem Objekt und zwei Projektionen \emph{Faserprodukt}, falls:
\begin{enumerate}
\item $f.\fib(f,g)_1 = g.\fib(f,g)_2$
\item Für jedes weitere Paar von Morphismen $\theta_1 : M \to X, \theta_2 : M \to X$ erhalten wir einen eindeutigen Pfeil $\mu : M \to X\times_Z Y$, sodass $\theta_i = \fib(f,g)_i . \mu$ für $i=1,2$.
\end{enumerate}
Dies definiert ein Faserprodukt eindeutig bis auf eindeutigen Iso.
\end{defn}

\begin{prop} Sei $\mathcal C$ eine Kategorie, $f : X \to Z, g : Y\to Z$ Morphismen sowie $X\times_Z Y$ deren Faserprodukt. Ist $g: Y\to Z$ ein Mono oder Iso, so ist auch die Projektion $\fib(f,g)_1: X\times_Z Y \to X$ ein Mono bzw. ein Iso.
\end{prop}

\begin{korr}
Sei $\mathcal A$ eine abelsche Kategorie und
$X_1 \hookr X$ und $X_2 \hookr X$ Monos in $\mathcal A$. Dann ist der induzierte Pfeil $X_1 \times_X X_2 \to X$ auch Mono als Komposition $X_1\times_X X_2 \hookr X_2 \hookr X$ von Monos.
\end{korr}

\begin{prop} Sei $\mathcal A$ eine abelsche Kategorie, $f : X \to Z, g : Y\to Z$ Morphismen. Ist $g: Y\to Z$ ein Epi, so ist auch die Projektion $\fib(f,g)_1: X\times_Z Y \to X$ ein Epi.
\end{prop}

\begin{satz}[Staffelung]
Seien $f: X\to Y,g : Y\to Z,h:W\to Z$ Pfeile und (bla, Existenzaussage damit alle folgenden Faserprodukte definiert sind...). Dann ist $X\times_{g.f,Z,h} W = X\times_{f,Y,\fib(g,h)_1} (Y\times_{g,Z,h} W)$ und es gelten
\begin{itemize}
\item $\fib(g.f,h)_1 = \fib(f,\fib(g,h)_1)_1$
\item $\fib(g.f,h)_2 = \fib(g,h)_2 . \fib(f,\fib(g,h)_1)_2$
\end{itemize}
\end{satz}
