\documentclass[3p]{elsarticle}

\title{Handbuch: Rechnen in Abelschen Kategorien}
\author{Long Huynh Huu}
\ead{long.huynh-huu@tum.de}


\usepackage{amsfonts,amsmath,amssymb}
\usepackage[utf8]{inputenc}
\usepackage[ngerman]{babel}


\newtheorem{satz}{Satz}[section]
\newtheorem{prop}{Proposition}[section]
\newtheorem{lemm}{Lemma}[section]
\newtheorem{korr}{Korollar}[section]
\newdefinition{defn}{Definition}[section]
\newdefinition{bem}{Bemerkung}[section]
\newproof{bew}{Beweis}

\newcommand\hookr\hookrightarrow
\newcommand\epicr\twoheadrightarrow
\newcommand\monicr\hookrightarrow
\DeclareMathOperator\im{im}
\DeclareMathOperator\id{id}
\DeclareMathOperator\coker{coker}
\DeclareMathOperator\coim{coim}
\DeclareMathOperator\fib{fib}
\DeclareMathOperator\cofib{cofib}
\DeclareMathOperator\Bild{Bild}
\DeclareMathOperator\Kern{Kern}

\DeclareMathOperator\para{par}
\DeclareMathOperator\fork{fork}
\DeclareMathOperator\join{join}


\begin{document}

\maketitle
\tableofcontents


\begin{abstract}
Eine Sammlung elementarer Aussagen zu Konstruktionen in abelschen Kategorien.
\end{abstract}

\section{Produkte und Coprodukte}

%%%%%%%%%%%
\subsection{Für allgemeine Kategorien}
%%%%%%%%%%%

Anders als im Titel erwähnt werde ich in diesem Abschnitt keine Coprodukte erwähnen -- es gelten einfach die dualen Aussagen.

Sei $\mathcal C$ eine Kategorie, $(Y_i)_{i\in I}$ eine Familie von Objekten und es existiere ein Produkt $\prod_i Y_i := \prod_{i\in I} Y_i$ mit Projektionen $\pi_j : \prod_i Y_i \to Y_j$.

\begin{prop} Die Projektionen sind Epis.
\end{prop}

Sei $X\in \mathcal C$ und $f_i : X \to Y_i$ Pfeile. Nach der UniE des Produktes wird ein eindeutiger Pfeil $X \to \prod_i Y_i$ induziert, den ich mit $\fork_i f_i$ bezeichne, sodass für alle $i\in I$ gilt: $f_i = \pi_j . \fork_i f_i$. Alternativ kann man auch $\fork(f_1,...,f_n)$ schreiben.

Im Falle von Coprodukten schreibt man $\join_i f_i$ bzw. $\join(f_1,...,f_n)$.

\begin{prop} Sei $u : W \to X$ ein Pfeil. Dann gilt $(\prod_i f_i) . u = \prod (f_i . u): W \to \prod_i Y_i$.
\end{prop}

% TODO: alles für summen

%%%%%%%%%%%
\subsection{In Abelschen Kategorien}
%%%%%%%%%%%

Sei nun $\mathcal A$ eine abelsche Kategorie, $X,X_1,..X_n$ sowie $Y,Y_1,..Y_n$ Objekte und $f_i : X_i \to Y_i$ (für $i = 1,..n$) Pfeile.
Für die direkten Summen $\bigoplus_i X_i$ bzw. $\bigoplus_i Y_i$ bezeichne die Inklusionen und Projektionen mit $\iota^X_j : X_j \to \bigoplus_i X_i$ und $\pi^X_j : \bigoplus_i X_i \to X_j$ bzw. $\iota^Y_j : Y_j \to \bigoplus_i Y_i$ und $\pi^Y_j : \bigoplus_i Y_i \to Y_j$.

\begin{prop} Die $f_i$ induzieren eindeutig einen Pfeil $\para_i f_i : \bigoplus_i X_i \to \bigoplus_i Y_i$ mit $f_j = \pi^Y_j . \para_i f_i . \iota^X_j$ für alle $j$.
\end{prop}

\begin{prop} Sind die $f_i$ Monos, Epis oder Isos, so entsprechend auch $\para_i f_i$.
\end{prop}

%%%%%%%%%%%
\subsection{Fork, Join, Parallel}
%%%%%%%%%%%

%TODO. Namenswahl erklären

Fork, Join und Par(allel) von Pfeilen hängen wie folgt zusammen:

\begin{prop}
Seien mit $f_i : V\to W_i, g_i : W_i \to X_i, h_i : X_i \to Y_i, k : Y_i \to Z$ für $i=1,...,n$ Pfeile bezeichnet.
Es gelten
\begin{enumerate}
\item $\fork(g_1.f_1,...) = \para(g_1,...).\fork(f_1,...)$
\item $\para(h_1.g_1,...) = \para(h_1,...).\para(g_1,...)$
\item $\join(k_1.h_1,...) = \join(k_1,...).\para(h_1,...)$
\end{enumerate}
\end{prop}

\section{Faserprodukte (und Fasersummen)}

Ein wenig Notation: Ich bezeichne für Pfeile $f : X\to Z, g: Y\to Z$ die Projektionen des Faserprodukts als $\fib_1(f,g) : X\times_Z Y \to X$ und $\fib_2(f,g) : X\times_Z Y \to Y$. Weiter definiere $\fib(f,g) := f.\fib_1(f,g) = g.\fib_2(f,g)$.
Es kommutieren also notwendigerweise die Argumente $\fib(f,g) = \fib(g,f)$.
Für Fasersummen ersetze ich $\fib$ durch $\cofib$.
Um etwas mit dieser Notation vertraut zu werden verwende ich sie gleich in der folgenden Definition.

\begin{defn}[Universelle Eigenschaft des Faserprodukts]
Seien $f:X\to Z, g:Y\to Z$ Pfeile, dann heißt das Tripel $(X\times_Z Y, \fib_1(f,g), \fib_2(f,g))$ bestehend aus einem Objekt und zwei Projektionen \emph{Faserprodukt}, falls:
\begin{enumerate}
\item $f.\fib_1(f,g) = g.\fib_2(f,g)$
\item Für jedes weitere Paar von Morphismen $\theta_1 : M \to X, \theta_2 : M \to X$ erhalten wir einen eindeutigen Pfeil $\mu : M \to X\times_Z Y$, sodass $\theta_i = \fib_i(f,g) . \mu$ für $i=1,2$.
\end{enumerate}
Dies definiert ein Faserprodukt eindeutig bis auf eindeutigen Iso.
\end{defn}

\begin{prop} Sei $\mathcal C$ eine Kategorie, $f : X \to Z, g : Y\to Z$ Morphismen sowie $X\times_Z Y$ deren Faserprodukt. Ist $g: Y\to Z$ ein Mono oder Iso, so ist auch die Projektion $\fib_1(f,g): X\times_Z Y \to X$ ein Mono bzw. ein Iso.
\end{prop}

\begin{prop}
Ein Spezialfall: $\fib_1(f,\id_Y) = f$ und $\fib_2(f,\id_Y) = \id_X$ für $f : X\to Y$.
\end{prop}

\begin{korr}
Sei $\mathcal A$ eine abelsche Kategorie und
$X_1 \hookr X$ und $X_2 \hookr X$ Monos in $\mathcal A$. Dann ist der induzierte Pfeil $X_1 \times_X X_2 \to X$ auch Mono als Komposition $X_1\times_X X_2 \hookr X_2 \hookr X$ von Monos.
\end{korr}

\begin{prop} Sei $\mathcal A$ eine abelsche Kategorie, $f : X \to Z, g : Y\to Z$ Morphismen. Ist $g: Y\to Z$ ein Epi, so ist auch die Projektion $\fib_1(f,g): X\times_Z Y \to X$ ein Epi.
\end{prop}

\begin{satz}[Staffelung]
Seien $f: X\to Y,g : Y\to Z,h:W\to Z$ Pfeile und (bla, Existenzaussage damit alle folgenden Faserprodukte definiert sind...). Dann ist $X\times_{g.f,Z,h} W = X\times_{f,Y,\fib_1(g,h)} (Y\times_{g,Z,h} W)$ und es gelten
\begin{itemize}
\item $\fib_1(g.f,h) = \fib_1(f,\fib_1(g,h))$
\item $\fib_2(g.f,h) = \fib_2(g,h) . \fib_2(f,\fib_1(g,h))$
\end{itemize}
\end{satz}

\begin{prop}[Kommutativität]
Seien $f : X\to Z, g:Y\to Z$ Pfeile. Dann gilt
\[ X\times_Z Y = Y\times_Z X \]
mit $\fib_1(f,g) = \fib_2(g,f)$ und $\fib_2(f,g) = \fib_1(g,f)$.
An dieser Stelle ist es sinnvoll nochmal zu erwähnen, dass $\fib(f,g) = \fib(g,f)$.
\end{prop}

\begin{prop}[Assoziativität]
Seien $f : W \to Z, g: X \to Z, h : Y\to Z$ Pfeile und $\{ \alpha, \beta, \gamma \} = \{ f,g,h\}$ eine beliebige Permutation dieser Pfeile. Dann gelten
\[ \fib(\fib(f,g),\fib(f,h)) = \fib(\fib(\alpha,\beta),\fib(\alpha,\gamma)). \]
Genauer haben wir sogar
\[ \fib_1(\fib_1(f,g),\fib_1(f,h)) = \fib_1(\fib_1(g,f),\fib_1(g,h)) \]
und schließlich gilt auch noch
\[ \fib(\fib(f,g),h) = \fib(f,\fib(g,h)) \]
\end{prop}

\begin{korr}[Faserproduktwürfel]
Im folgenden werden wir etwas Vorstellungskraft brauchen: Mit $E_{000},E_{001},...,E_{111}$ werden wir Objekte bezeichnen, die auf den Ecken eines Einheitsquadrats $[0,1]^3$ sitzen, nämlich an den Punkten $(0,0,0),(0,0,1),...(1,1,1)$.
Gegeben seien drei Pfeile $f : E_{100} \to E_{110}, g : E_{111} \to E_{110}, h : E_{010} \to E_{110}$ gegeben. Die Behauptung ist nun, dass es möglich ist, jeder verbleibenden Kante genau einen Pfeil und jeder verbleibenden Ecke ein Objekt zuzuordnen, sodass auf jeder Seite des Würfels ein Faserproduktdiagramm steht.

%TODO
[Hier kommt eine nette Illustration hin]
\end{korr}

\begin{prop}
Für $f: X\to Z, g:Y\to Z, h:Z\to Z'$ existiert ein eindeutiger Pfeil $\kappa : X\times_{f,Z,g} Y \to X\times_{h.f,Z,h.g} Y$ mit $\fib_i(f,g) = \fib_i(h.f,h.g).\kappa$ für $i=1,2$. Ist auch noch $h$ ein Mono, dann ist $\kappa$ ein Iso.
\end{prop}


\section{Kerne, Cokerne, Bilder und Cobilder}

Wir sind nun in einer abelschen Kategorie $\mathcal A$.

%%%%%%%%%%%
\subsection{Kerne (und Cokerne)}
%%%%%%%%%%%


\begin{prop}\label{mono-kern-0}
Ein Pfeil $\beta : Y \monicr Z$ ist genau dann ein Mono, wenn $\ker(\beta) = 0_{0,Y}$.

Ein Pfeil $\alpha : X \epicr Y$ ist genau dann ein Epi, wenn $\coker(\alpha) = 0_{Y,0}$.
\end{prop}
\begin{bew}
Ist $\beta$ ein Mono, dann gilt für jeden Pfeil $u$ mit $\beta.u = 0$, dass $\beta.u = 0 = \beta.0$, d.h. $u = 0$, was bedeutet, dass $u$ durch das Nullobjekt faktorisiert. 
Umgekehrt sei $\ker(\beta) = 0_{0,Y}$ und $u,v$ Pfeile mit $\beta.u = \beta.v$. Dann folgt $0 = \beta.(u-v)$, d.h. $u-v$ faktorisiert durch den Kern, aber der war Null. Damit ist $u-v = 0$ und dementsprechend $u=v$.
Der Beweis für Epis verläuft analog.
\end{bew}

\begin{prop}\label{ker-examples}
Für $X,Y \in \mathcal A$ gelten
\begin{enumerate}
\item $\ker(\id_Y) = 0_{0,Y}$
\item $\coker(\id_X) = 0_{X,0}$
\item $\ker(0_{X,Y}) = \id_X$
\item $\coker(0_{X,Y}) = \id_Y$
\end{enumerate}
\end{prop}
\begin{bew}
$\id_Y$ ist ein Mono und $\id_X$ ein Epi. Aus Proposition \ref{mono-kern-0} folgen die ersten beiden Gleichungen. Die anderen beiden folgen direkt aus den UniEs von Kern und Cokern.
\end{bew}



\begin{prop}\label{ker-is-fibre}
Man kann Kerne als Faserprodukte ausdrücken. Sei $f : X\to Y$, dann gilt
\[\ker f = \fib_1(f,0_{0,Y}) \]
Ist weiter $g : Y\to Z$ ein Pfeil, haben wir sogar
\[ \ker(g.f) = \fib_1(g,\ker f) \]
\end{prop}
\begin{bew}
Die UniEs von Kern und Faserprodukt stimmmen in diesem Fall überein, daher $\ker(f) = \fib_1(f,0_{0,Y})$.
Aus der Staffelung von Faserprodukten folgt
\[\ker(g.f) = \fib_1(g.f,0_{0,Y}) = \fib_1(g,\fib_1(f,0_{0,Y})) = \fib_1(g,\ker f)\]
\end{bew}

%%%%%%%%%%%
\subsection{Bilder (und Cobilder)}
%%%%%%%%%%%

\begin{defn}
Sei $f : X\to Y$ ein Pfeil, dann bezeichnet man den Kern des Cokerns $\ker(\coker f) =: \im f$ als Bild von $f$
und den Cokern des Kerns $\coker(\ker f) =: \coim(f)$ als Cobild von $f$.
\end{defn}

\begin{prop}\label{unie-im}
Sei $f:X\to Y$ ein Pfeil, dann erfüllt jedes Bild $\im f =: g : B \to Y$ folgende UniE:
Es gibt einen Pfeil $h : X \to B$, sodass $g.h = f$. Weiters gibts für jede Faktorisierung $f = m . e : X \to B' \to Y$ mit $m$ Mono einen eindeutigen Pfeil $\gamma : B \to B'$, sodass $g = m.\gamma$.
\end{prop}

\begin{satz}\label{epi-mono-proj}
Seien $\alpha : X \to Y, \beta : Y \monicr Z$ Pfeile und $\beta$ Mono.
Dann gelten
\[ \im(\beta.\alpha) = \beta.\im \alpha
\quad\text{und}\quad
\coim(\beta.\alpha) = \coim \alpha.\]
Nun seien $\alpha' : X \epicr Y, \beta' : Y \to Z$ Pfeile und $\alpha'$ Epi.
Dann gelten
\[ \im(\beta'.\alpha') = \im \beta'
\quad\text{und}\quad
 \coim(\beta'.\alpha') = \coim \beta'.\alpha'.\]
\end{satz}
\begin{bew}
Wir prüfen die UniE des Bildes (Proposition \ref{unie-im}).
Nach UniE von $\alpha$ existiert ein Pfeil $e$ mit $\im \alpha.e = \alpha$.
Dadurch erhalten wir $\beta.\im \alpha.e = \beta.\alpha$.
Seien nun $a,b$ Pfeile, $b$ ein Mono, sodass $b.a = \beta . \alpha$.


Die verbleibenden Aussagen sind lediglich die dualen Aussagen des bereits Bewiesenen.
\end{bew}


\begin{korr}
Für $X,Y \in \mathcal A$ gelten
\begin{enumerate}
\item $\im \id_X = \id_X$
\item $\coim \id_Y = \id_Y$
\item $\im 0_{X,Y} = 0_{0,Y}$
\item $\coim 0_{X,Y} = 0_{X,0}$
\end{enumerate}
\end{korr}
\begin{bew}
Folgt direkt aus Satz \ref{epi-mono-proj}.
\end{bew}

\begin{korr}
Seien $\alpha : X \epicr Y, \beta : Y \monicr Z$, $\alpha$ Epi und $\beta$ Mono, dann gilt $\im(\beta.\alpha) = \beta$. Insbesondere gelten $\im \beta = \beta$ und $\im \alpha = \id_Y$.

Dual dazu: $\coim(\beta.\alpha) = \alpha, \coim \alpha = \alpha, \coim \beta = \id_Y$.
\end{korr}
\begin{bew}
$\im(\beta.\alpha) = \im \beta = \beta.\im \id = \beta$. Für die anderen beiden Aussagen setzte jeweils $\alpha = \id_Y, X=Y$ bzw. $\beta = \id_Y, Y=Z$.

Für die duale Aussage dualisiere obigen Beweis.
\end{bew}
\begin{bem}
Umgekehrt folgt aus $\im\beta' = \beta'$ für einen Pfeil $\beta'$, dass $\beta'$ ein Mono ist, d.h.:
Es ist $\beta$ genau dann Mono, wenn $\im\beta = \beta$
\end{bem}

\begin{korr}
Monos sind Kerne und Kerne sind Monos.
\end{korr}
\begin{bew}
Ist $\beta$ ein Mono, so gilt $\beta = \im\beta = \ker(\coker\beta)$.
Jeder Kern $\ker(f)$ ist Mono, denn nach Proposition \ref{ker-is-fibre} ist $\ker f = \fib_1(f,0_{0,Y})$.
Der Pfeil $0_{0,Y}$ ist ein Mono wegen Proposition \ref{mono-kern-0} und Proposition \ref{ker-examples} und damit folgt aus Proposition \ref{fibre-mono}, dass auch $\fib_1(f,0_{0,Y})$ Mono ist.
\end{bew}

\begin{korr}[Idempotenz des Bildes/Cobildes]
Für jeden Pfeil $f$ gelten
\[ \im(\im f) = \im f \quad\text{und}\quad \coim(\coim f) = \coim f \]
\end{korr}
\begin{bew}
Wende Satz \ref{epi-mono-proj} an.
\end{bew}

\begin{bem}[Epi-Mono-Faktorisierung]
Für jeden Pfeil $f$ einer abelschen Kategorie gibt es per Axiom einen Isomorphismus $\tau$, sodass $f = \im f.\tau.\coim f$.
Man kann also $\im$ als Projektion eines Pfeils $f$ mit Epi-Mono-Faktorisierung $\mu.\varepsilon$ auf seine Mono-Komponente und $\coim$ als Projektion auf seine Epi-Komponente verstehen, denn $\im f = \mu$ und $\coim f = \varepsilon$.

\end{bem}

\begin{prop} \label{im-join-im-elim}
Es gilt $\im\join(\im(g),h) = \im\join(g,h)$ für alle Pfeile $g:X\to Z,h : Y\to Z$.
\end{prop}
\begin{bew}
$\im\join(\im g,h) = \im(\join(\im g,h).\para(\coim g,\id_Y)) = \im\join(g,h)$
\end{bew}

\section{2-Komplexe}

%TODO: Alles auf 2-Komplexe ummünzen

\begin{defn}
Sei $\mathcal C$ eine Kategorie und $p : X_2 \to X_1, q: Y_2 \to Y_1$ Pfeile, welche wir im folgenden als Paare bezeichnen und $p =: (X_1,X_2), q =: (Y_1,Y_2)$ schreiben.
Wir nennen jedes Paar von Pfeilen $f_1 : X_1 \to Y_1, f_2 : X_2 \to Y_2$ einen \emph{Pfeil von Paaren $(X_1,X_2),(Y_1,Y_2)$} und schreiben $(f_1,f_2) : (X_1,X_2) \to (Y_1,Y_2)$, falls gilt folgende Verträglichkeit erfüllt ist:
\[ f_1 . p = q . f_2 \]
\end{defn}

\begin{prop}
Sind $(f_1,f_2) : (X_1,X_2) \to (Y_1,Y_2)$ und $(g_1,g_2) : (Y_1,Y_2) \to (Z_1,Z_2)$ Pfeile von Paaren, dann auch $(g_1.f_1, g_2.f_2) =: (g_1,g_2).(f_1,f_2)$.
Insbesondere gilt $(f_1,f_2).(\id_{X_1},\id_{X_2}) = (f_1,f_2) = (\id_{Y_1},\id_{Y_2}).(f_1,f_2) $.
\end{prop}
\begin{bew}
Seien $x : X_2 \to X_1, y:Y_2 \to Y_1, z : Z_2 \to Z_2$ die zu den Paaren assoziierten Pfeile.
Dann gilt $g_1.f_1 . x = g_1.y.f_2 = z.g_2.f_2$.
\end{bew}

\begin{bem}
Damit erhalten wir für eine Kategorie $\mathcal C$ eine Kategorie $Paar(\mathcal C)$ mit Paaren (also Elementen aus $Mor(\mathcal C)$) als Objekte und Pfeile von Paaren als Pfeile.
Weiter können wir die vollen Unterkategorien $MonoPaar(\mathcal C)$, deren Objekte die Monos von $Mor(\mathcal C)$ sind, und $EpiPaar(\mathcal C)$, deren Objekte die Epis von $Mor(\mathcal C)$ sind, definieren.
\end{bem}

\begin{prop}
Sei $\mathcal A$ eine abelsche Kategorie und $x' : X'\monicr X$ sowie $y' : Y' \monicr Y$ Monos.
Sei nun $(f,f') : (X,X')\to (Y,Y')$ ein Pfeil von Paaren.
Dann existiert genau ein Pfeil $\bar f : X/X' \to Y/Y'$, sodass $\bar f . \coker(x') = \coker(y') . f$, d.h. $(\bar f, f) : (X/X', X) \to (Y/Y', Y)$ ist ein Pfeil von Paaren.

Sind umgekehrt $\bar x : X \epicr \bar X, \bar y : Y \epicr \bar Y$ Epis und ein Pfeil von Paaren $(\bar f, f)  : (\bar X,X) \to (\bar Y, Y)$ gegeben, so erhalten wir eindeutig einen Pfeil von Paaren $(f,f') : (X,\Kern(\bar x)) \to (Y,\Kern(\bar y))$.
\end{prop}
\begin{bew}
Es gilt $(\coker(y').f).x' = 0_{X,Y/Y'}.x'$:
\[
\coker(y').f.x' = \coker(y').y'.f' = 0_{X',Y/Y'}
\]
Nach UniE von $\coker(x')$ gibt es genau einen Pfeil $\bar f : X/X' \to Y/Y'$ mit
\[ \bar f . \coker(x') = \coker(y') . f \]

Für die Umkehrung betrachte $\bar y.(f.\ker(\bar x)) = \bar f . \bar x . \ker(\bar x) = 0_{X',\bar Y}$, daher existiert nach UniE von $\ker(\bar y)$ genau ein Pfeil $f' : \Kern(\bar x) \to \Kern(\bar y)$ mit 
\[ f.\ker(\bar x) = \ker(\bar y).f' \]
\end{bew}

\begin{bem}
Erinnern wir uns mal daran, dass in abelschen Kategorien Monos genau die Kerne und Epis genau die Cokerne sind. Daher gibt uns die obige Proposition eine Äquivalenz der Kategorien $MonoPaar(\mathcal A)$ und $EpiPaar(\mathcal A)$ für eine abelsche Kategorie $\mathcal A$.
\end{bem}

\begin{defn}[Objektordnung]
Sei $\mathcal C$ eine Kategorie.
Wir definieren eine Relation $\leq$ auf Unterobjekten von einem festen $X \in \mathcal C$ wie folgt: Für Unterobjekte $x_1 : X_1 \monicr X, x_2 : X_2 \monicr X$ sei
\[ x_1 \leq x_2 \Leftrightarrow \exists (j:  X_1 \to X_2) : x_1 = x_2 . j \]
Existiert solch ein $j$, so ist es notwendigerweise ein Mono.

Auf Quotientenobjekten können wir eine Relation $\leq^{op}$ definieren:
Für Quotientenobjekte $x^1 : X \epicr X^1, x^2 : X \epicr X^2$ sei
\[ x^1 \leq^{op} x^2 \Leftrightarrow \exists (k : X^2 \to X^1) : x^1 = k . x^2 \]
Jedes solche $k$ muss ein Epi sein.
\end{defn}

\begin{lemm}
Mit $\leq$ und $\leq^{op}$ haben wir eine Ordnungsrelationen auf der Menge der Unterobjekte, bzw. Quotientenobjekte von $X$ definiert.
\end{lemm}
\begin{bew}
Ich führe den Beweis nur für $\leq$:
Es seien $x_i : X_i \monicr X_i, i=1,2,3$ Unterobjekte.
Die Relation ist reflexiv, denn $x_1 = x_1.\id_{X_1}$; sie ist transitiv, denn für $x_1 \leq x_2 \leq x_3$ erhalten wir Pfeile $j,j'$ mit $x_1 = x_2.j$ und $x_2 = x_3.j'$, also insgesamt $x_1 = x_3.(j'.j)$.
Für Antisymmetrie nehmen $x_1 \leq x_2$ und $x_2 \leq x_1$ an und erhalte Pfeile $j,j'$ mit $x_1 = x_2.j$ sowie $x_2 = x_1.j'$. Da $x_1,x_2$ Monos sind folgt dass $j$ ein Iso ist:
\[ x_1 . \id_{X_1} = x_1.(j'.j) \implies j'.j = \id_{X_1} \]
\[ x_2 . \id_{X_2} = x_2.(j.j') \implies j.j' = \id_{X_2} \]
Damit definieren $x_1$ und $x_2$ das gleiche Unterobjekt.

\end{bew}

\begin{korr}\label{antitonic}
Obige Äquivalenz von Kategorien ist ordnungserhaltend, womit folgendes gemeint ist: Sind $X_1 \monicr X, X_2 \monicr X$ Unterobjekte von $X$, dann gilt
\[ X_1 \leq X_2 \Leftrightarrow X/X_2 \leq^{op} X/X_1, \]
\end{korr}
\begin{bew}
\begin{align*}
                 & X_1 \leq X_2
\\\Leftrightarrow& \exists j : (\id_X,j) : (X,X_1) \to (X,X_2) \text{ ist Pfeil von 2-Komplexen}
\\\Leftrightarrow& \exists k : (k,\id_X) : (X/X_1,X) \to (X/X_2,X) \text{ ist Pfeil von 2-Komplexen}
\\\Leftrightarrow& X/X_2 \leq^{op} X/X_1
\end{align*}
\end{bew}


\begin{prop}
Seien 2-Komplexe $X' \monicr X$ und $Y' \monicr Y$ sowie $f : X \to Y$ gegeben. Dann existiert genau ein $f' : X' \to Y'$, sodass $(f,f') : (X,X') \to (Y,Y')$ ein Pfeil von 2-Komplexen ist, falls $f(X') \leq Y'$.

Dual dazu: Sind 2-Komplexe $X \epicr X/X'$ und $Y \epicr Y/Y'$ sowie $f : X\to Y$ gegeben. Dann existiert genau ein $\bar f : X/X' \to Y/Y'$, sodass $(\bar f, f) : (X/X',X) \to (Y/Y',Y)$ ein Pfeil von 2-Komplexen ist, falls $X' \leq f^{-1}(Y')$.
\end{prop}

\section{Rolands Vierzehn}

Hier sind die 14 Rechenregeln, die uns der Prophet Roland L. auf Steintafeln gebracht hat. Das Studium dieser Aussagen über abelsche Kategorien hat dieses Handbuch motiviert.

Im folgenden sei $\mathcal A$ eine abelsche Kategorie und mit $X,
X_1 \overset{x_1}\monicr X,
X_2 \overset{x_2}\monicr X,
X'  \overset{x'}\monicr X,
Y,
Y_1 \overset{y_1}\monicr Y,
Y_2 \overset{y_2}\monicr Y,
Y'  \overset{y'}\monicr Y,
Z,
Z'  \overset{z'}\monicr Z$
seien Objekte und Unterobjekte dieser Kategorie gemeint.
Weiter seien $f:X\to Y, g:Y\to Z$ Pfeile.

\begin{defn}[Objektnotation]
Wir definieren
\begin{itemize}
\item $X_1 \cap X_2 := X_1\times_Z X_2$ als Unterobjekt von $X$ via $\fib(x_1, x_2)$.
\item $X_1+X_2 = \Bild(X_1 \oplus X_2 \to X)$ als Unterobjekt von $X$ via $\im(\coprodop(x_1,x_2))$.
\item $f^{-1}(Y') = X \times_{f,Y,y'} Y'$ als Unterobjekt von $X$ via $\fib_1(f,y')$
\item $f(X') = \Bild(f.x')$ als Unterobjekt von $Y$ via $\im(f.x')$.
\end{itemize}
\end{defn}

\begin{bem}
Ohne Beweis habe ich oben angenommen, dass die vorkommenden Inklusionspfeile Monos sind, hier eine kurze Begründung: etc.pp. %TODO
\end{bem}

\begin{prop}[Reflexionsprinzip]
Unterobjekte und Quotientenobjekte stehen in enger Beziehung zueinander:
\begin{enumerate}
\item $\cofib(\coker(x_1), \coker(x_2)) = \coker(\im(\coprodop(x_1,x_2)))$, (d.h. $X/X_1 \cap X/X_2 = X/(X_1+X_2)$ mit Schnitt in $\mathcal A^{op}$)
\item $\coim(\prodop(\coker(x_1),\coker(x_2))) = \coker(\fib(x_1,x_2))$ (d.h. $X/X_1 + X/X_2 = X/(X_1\cap X_2)$ mit Summe in $\mathcal A^{op}$)
\item $\coim(\coker(y').f) = \coker(\fib_1(f,y'))$ (d.h. $f^{op}(Y/Y') = X/f^{-1}(Y')$ mit Bild in $\mathcal A^{op}$)
\item $\cofib_1(f,\coker(x')) = \coker(\im(f.x'))$ (d.h. $(f^{op})^{-1}(X/X') = Y/f(X')$ mit Urbild in $\mathcal A^{op}$)
\end{enumerate}
Siehe (Blatt 1, Warmup)
\end{prop}
\begin{bew}
%TODO
Beweise mich, ich bin nicht schwierig! (nur nervig!)
\end{bew}

\begin{lemm}[Regel (a)] $\quad$ % erzwinge Zeilenumbruch
\begin{enumerate}
\item $\im(0_{0,X}) = 0_{0,X}$ (d.h. $f(0) = 0$)
\item $\im(f.\id_X) = \im(f)$ (d.h. $f(X) = X$)
\item $\fib_1(f,0_{0,Y}) = \ker(f)$ (d.h. $f^{-1}(0) = \Kern(f)$)
\item $\fib_1(f,\id_Y) = \id_X$ (d.h. $f^{-1}(Y) = X$)
\end{enumerate}
\end{lemm}

\begin{lemm}[Regel (b)]
Es gelten $X_1 \cap X_2 = X_2 \cap X_1$
sowie $X_1 + X_2 = X_2 + X_1$ als Unterobjekte.
\end{lemm}
\begin{bew}
%TODO
verwende das Reflexionsprinzip etc.
\end{bew}

\begin{lemm}[Regel (c)]
Es gelten $X' \cap (X_1 \cap X_2) = (X' \cap X_1) \cap X_2$ 
sowie $X' + (X_1 + X_2) = (X' + X_1) + X_2$ als Unterobjekte.
\end{lemm}
\begin{bew}
%TODO
Formuliere universelle Eigenschaft für Dreifachschnitte bzw. Dreifachsummen, etc.
\end{bew}

\begin{lemm}[Regel (d)] $\quad$ % Zeilenumbruch
% TODO: Prüfe auf Richtigkeit!
\begin{enumerate}
\item $\fib_1(g.f, z') = \fib_1(f, \fib_1(g, z'))$ (d.h. $(g.f)^{-1}(Z') = f^{-1}(g^{-1}(Z'))$)
\item $\im(g.f.x') = \im(g.\im(f.x'))$ (d.h. $(g.f)(X') = g(f(X'))$)
\end{enumerate}
\end{lemm}
\begin{bew}
Die erste Gleichung ist nur ein Spezialfall der Staffelung von Faserprodukten.

Die zweite Gleichung folgt unmittelbar, wenn wir die Epi-Mono-Zerlegung $f.x' = \im(f.x').\varepsilon$ einsetzen:
\[\im(g.f.x') = \im(g.\im(f.x').\varepsilon) = \im(g.\im(f.x')) \]
\end{bew}

\begin{lemm}[Regel (e)]
\[
\im(\fib(f,y')) = \fib(\im(f),y')
\]
und
\[
\fib_1(f,\im(f.x')) = \im(\coprodop(x', \ker(f)))
\]
In Objektnotation heißt dies: $f(f^{-1}(Y')) = Y' \cap \Bild(f)$, bzw. $f^{-1}(f(X')) = X' + \Kern(f)$.
\end{lemm}
\begin{bew}
Zur ersten Gleichung:
Sei eine Epi-Mono-Zerlegung $f=\im(f).\varepsilon$ gegeben.
Dann steht auf der linken Seite $\im(f.\fib_1(f,y')) = \im(\im(f).\varepsilon.\fib_1(f,y')) = \im(f).\im(\varepsilon.\fib_1(f,y'))$ und auf der rechten Seite $\im(f).\fib_1(\im(f),y')$. Es reicht also, zu zeigen, dass
\[ \im(\varepsilon.\fib_1(f,y')) = \fib_1(\im(f),y'). \]
Dazu schreiben wir erst $\varepsilon.\fib_1(f,y')$ um:
\begin{align*}
&\varepsilon.\fib_1(f,y')
\\=& \varepsilon.\fib_1(\im(f).\varepsilon,y')
\\=& \varepsilon.\fib_1(\varepsilon,\fib_1(\im(f),y'))
\\=& \fib_1(\im(f),y').\fib_2(\varepsilon,\fib_1(\im(f),y'))
\end{align*}
Der letzte Term ist eine Komposition eines Monos mit einem Epi, d.h. nach Anwendung von $\im$ überlebt nur der Mono:
\[ \im(\varepsilon.\fib_1(f,y')) = \fib_1(\im(f),y') \]
was zu zeigen war.

% Nun zur zweiten Gleichung:
% Sei $f.x' = \im(f.x').\varepsilon$ eine Epi-Mono-Faktorisierung.
% Dann faktorisiert $f.x' : X' \to Y$ über einen Pfeil $\theta : X' \to f^{-1}(f(X'))$ durch $\fib_1(f,\im(f.x'))$.
% Ebenso gilt $f.\ker(f) = \im(f.x').0$, d.h. $f.\ker(f) : \Kern(f) \to Y$ faktorisiert über einen Pfeil $\tau : \Kern(f) \to f^{-1}(f(X'))$ durch $\fib_1(f,\im(f.x'))$.
% Damit haben wir
% \begin{align*}
% \im(\coprodop(x',\ker(f)))
% =& \im(\coprodop(\fib_1(f,\im(f.x')).\theta,\fib_1(f,\im(f.x')).\tau))
% \\=& \im(\fib_1(f,\im(f.x')).\coprodop(\theta,\tau))
% \\=& \fib_1(f,\im(f.x')).\im(\coprodop(\theta,\tau))
% \end{align*}
% 
% % TODO: fix prooof
% Die Pfeile $\theta$ und $\tau$ induzieren einen Pfeil $\gamma : f^{-1}(f(X')) \to X'\oplus \Kern(f)$, $\gamma.x'$ und $\gamma.\ker(f)$ die Inklusionen der Komponenten $X', \Kern(f)$ in die direkte Summe sind.
% Nun prüfe die universelle Eigenschaft: Sei %TODO MÜÜÜÜDE
% MÜÜDEEE
% 
% Da $\fib_1(f,\im(f.x')) = \coprodop(x',\ker(f)).\gamma$ gilt und der linke Pfeil in Mono ist, ist auch $\gamma$ ein Mono.
% Nun gilt $\im(\coprod(x',\ker(f))) = \im(\coprodop(x',\ker(f)).\gamma) = \im(\fib_1(f,\im(f.x'))) = \fib_1(f,\im(f.x'))$, wie gewünscht.
% 
(Übung: Zeichne jeweils ein passendes kommutatives Diagramm. Das zweite könnte etwas wild werden.)
\end{bew}

\begin{lemm}[Regel (f)] $\quad$ %Zeilenumbruch
\begin{enumerate}
\item $\im(f.\im(\coprodop(x_1,x_2))) = \im(\coprodop(\im(f.x_1),\im(f.x_2)))$
\item $\fib_1(f,\fib(y_1,y_2)) = \fib_1(f,y_1) . \fib_1(\fib_1(f,y_1),\fib_1(f,y_2))$
\end{enumerate}
\end{lemm}

\begin{lemm}[Regel (g), Projektionsformel] $\quad$ %Zeilenumbruch
\begin{enumerate}
\item $\fib_1(f,\im(\coprodop(\im(f.x'),y'))) = \im(\coprodop(x',\fib_1(f,y')))$
\item $\im(f.\fib(\fib_1(f,y'),x')) = \fib(y',\im(f.x'))$
\end{enumerate}
\end{lemm}



\end{document}
