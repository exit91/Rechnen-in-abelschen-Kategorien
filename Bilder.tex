\section{Kerne, Cokerne, Bilder und Cobilder}

Wir sind nun in einer abelschen Kategorie $\mathcal A$.

%%%%%%%%%%%
\subsection{Kerne (und Cokerne)}
%%%%%%%%%%%


\begin{prop}
Für $X,Y \in \mathcal A$ gelten
\begin{enumerate}
\item $\ker(0_{X,Y}) = \id_X$
\item $\coker(0_{X,Y}) = \id_Y$
\item $\ker(\id_Y) = 0_{0,Y}$
\item $\coker(\id_X) = 0_{X,0}$
\end{enumerate}
\end{prop}

\begin{prop}
Ein Pfeil $\beta : Y \monicr Z$ ist genau dann ein Mono, wenn $\ker(\beta) = 0_{0,Y}$.

Ein Pfeil $\alpha : X \epicr Y$ ist genau dann ein Epi, wenn $\coker(\alpha) = 0_{Y,0}$.
\end{prop}

\begin{prop}
Man kann Kerne als Faserprodukte ausdrücken. Sei $f : X\to Y$, dann gilt
\[\ker(f) = \fib_1(f,0_{0,Y}) \]
\end{prop}

%%%%%%%%%%%
\subsection{Bilder (und Cobilder)}
%%%%%%%%%%%

\begin{defn}
Sei $f : X\to Y$ ein Pfeil, dann bezeichnet man den Kern des Cokerns $\ker(\coker(f)) =: \im(f)$ als Bild von $f$
und den Cokern des Kerns $\coker(\ker(f)) =: \coim(f)$ als Cobild von $f$.
\end{defn}

\begin{prop}
Sei $f:X\to Y$ ein Pfeil, dann erfüllt jedes Bild $\im(f) =: g : B \to Y$ folgende universelle Eigenschaft:
Es gibt einen Pfeil $h : X \to B$, sodass $g.h = f$. Weiters gibts für jede Faktorisierung $f = m . e : X \to B' \to Y$ mit $m$ Mono einen eindeutigen Pfeil $\gamma : B \to B'$, sodass $h = m.\gamma$.
\end{prop}

\begin{prop}
Für $X,Y \in \mathcal A$ gelten
\begin{enumerate}
\item $\im(\id_X) = \id_X$
\item $\coim(\id_Y) = \id_Y$
\item $\im(0_{X,Y}) = 0_{0,Y}$
\item $\coim(0_{X,Y}) = 0_{X,0}$
\end{enumerate}
\end{prop}


\begin{prop}
Seien $\alpha : X \to Y, \beta : Y \monicr Z$ Pfeile und $\beta$ Mono.
Dann gilt $\im(\beta.\alpha) = \beta.\im(\alpha)$.
\end{prop}


\begin{prop}
Seien $\alpha : X \epicr Y, \beta : Y \to Z$ Pfeile und $\alpha$ Epi.
Dann gilt $\im(\beta.\alpha) = \im(\beta)$.
\end{prop}

\begin{korr}
Seien $\alpha : X \epicr Y, \beta : Y \monicr Z$, $\alpha$ Epi und $\beta$ Mono, dann gilt $\im(\beta.\alpha) = \beta$. Insbesondere gelten $\im(\beta) = \beta$ und $\im(\alpha) = \id_Y$.
\end{korr}
\begin{bew}
$\im(\beta.\alpha) = \im(\beta) = \beta.\im(\id) = \beta$. Für die anderen Aussagen setzte jeweils $\alpha = \id_Y, X=Y$ bzw. $\beta = \id_Y, Y=Z$.
\end{bew}
\begin{bem}
Umgekehrt folgt aus $\im(\beta') = \beta'$ für einen Pfeil $\beta'$, dass $\beta'$ ein Mono ist, d.h.:
Es ist $\beta$ genau dann Mono, wenn $\im(\beta) = \beta$
\end{bem}

\begin{korr}
Monos sind Kerne und Kerne sind Monos.
\end{korr}
\begin{bew}
Ist $\beta$ ein Mono, so gilt $\beta = \im(\beta) = \ker(\coker(\beta))$.
Jeder Kern ist Mono nach Proposition soundso.
\end{bew}

\begin{korr}[Idempotenz des Bildes/Cobildes]
Für jeden Pfeil $f$ gilt $\im(\im(f)) = \im(f)$.
\end{korr}

\begin{bem}[Epi-Mono-Faktorisierung]
Für jeden Pfeil $f$ einer abelschen Kategorie gibt es per Axiom einen Isomorphismus $\tau$, sodass $f = \im(f).\tau.\coim(f)$.
\end{bem}

\begin{prop} \label{im-join-im-elim}
Es gilt $\im(\join(\im(g),h)) = \im(\join(g,h))$ für alle Pfeile $g:X\to Z,h : Y\to Z$.
\end{prop}
\begin{bew}
$\im(\join(\im(g),h)) = \im(\join(\im(g),h).\para(\coim(g),\id_Y)) = \im(\join(g,h))$
\end{bew}
