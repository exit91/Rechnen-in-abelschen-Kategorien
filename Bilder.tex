\section{Kerne, Cokerne, Bilder und Cobilder}

Wir sind nun in einer abelschen Kategorie $\mathcal A$.

%%%%%%%%%%%
\subsection{Kerne (und Cokerne)}
%%%%%%%%%%%


\begin{prop}\label{mono-kern-0}
Ein Pfeil $\beta : Y \monicr Z$ ist genau dann ein Mono, wenn $\ker(\beta) = 0_{0,Y}$.

Ein Pfeil $\alpha : X \epicr Y$ ist genau dann ein Epi, wenn $\coker(\alpha) = 0_{Y,0}$.
\end{prop}
\begin{bew}
Ist $\beta$ ein Mono, dann gilt für jeden Pfeil $u$ mit $\beta.u = 0$, dass $\beta.u = 0 = \beta.0$, d.h. $u = 0$, was bedeutet, dass $u$ durch das Nullobjekt faktorisiert. 
Umgekehrt sei $\ker(\beta) = 0_{0,Y}$ und $u,v$ Pfeile mit $\beta.u = \beta.v$. Dann folgt $0 = \beta.(u-v)$, d.h. $u-v$ faktorisiert durch den Kern, aber der war Null. Damit ist $u-v = 0$ und dementsprechend $u=v$.
Der Beweis für Epis verläuft analog.
\end{bew}

\begin{prop}\label{ker-examples}
Für $X,Y \in \mathcal A$ gelten
\begin{enumerate}
\item $\ker(\id_Y) = 0_{0,Y}$
\item $\coker(\id_X) = 0_{X,0}$
\item $\ker(0_{X,Y}) = \id_X$
\item $\coker(0_{X,Y}) = \id_Y$
\end{enumerate}
\end{prop}
\begin{bew}
$\id_Y$ ist ein Mono und $\id_X$ ein Epi. Aus Proposition \ref{mono-kern-0} folgen die ersten beiden Gleichungen. Die anderen beiden folgen direkt aus den UniEs von Kern und Cokern.
\end{bew}



\begin{prop}\label{ker-is-fibre}
Man kann Kerne als Faserprodukte ausdrücken. Sei $f : X\to Y$, dann gilt
\[\ker f = \fib_1(f,0_{0,Y}) \]
Ist weiter $g : Y\to Z$ ein Pfeil, haben wir sogar
\[ \ker(g.f) = \fib_1(g,\ker f) \]
\end{prop}
\begin{bew}
Die UniEs von Kern und Faserprodukt stimmmen in diesem Fall überein, daher $\ker(f) = \fib_1(f,0_{0,Y})$.
Aus der Staffelung von Faserprodukten folgt
\[\ker(g.f) = \fib_1(g.f,0_{0,Y}) = \fib_1(g,\fib_1(f,0_{0,Y})) = \fib_1(g,\ker f)\]
\end{bew}

%%%%%%%%%%%
\subsection{Bilder (und Cobilder)}
%%%%%%%%%%%

\begin{defn}
Sei $f : X\to Y$ ein Pfeil, dann bezeichnet man den Kern des Cokerns $\ker(\coker f) =: \im f$ als Bild von $f$
und den Cokern des Kerns $\coker(\ker f) =: \coim(f)$ als Cobild von $f$.
\end{defn}

\begin{prop}\label{unie-im}
Sei $f:X\to Y$ ein Pfeil, dann erfüllt jedes Bild $\im f =: g : B \to Y$ folgende UniE:
Es gibt einen Pfeil $h : X \to B$, sodass $g.h = f$. Weiters gibts für jede Faktorisierung $f = m . e : X \to B' \to Y$ mit $m$ Mono einen eindeutigen Pfeil $\gamma : B \to B'$, sodass $g = m.\gamma$.
\end{prop}

\begin{prop}\label{epi-mono-proj}
Seien $\alpha : X \to Y, \beta : Y \monicr Z$ Pfeile und $\beta$ Mono.
Dann gelten
\[ \im(\beta.\alpha) = \beta.\im \alpha
\quad\text{und}\quad
\coim(\beta.\alpha) = \coim \alpha.\]
Nun seien $\alpha' : X \epicr Y, \beta' : Y \to Z$ Pfeile und $\alpha'$ Epi.
Dann gelten
\[ \im(\beta'.\alpha') = \im \beta'
\quad\text{und}\quad
 \coim(\beta'.\alpha') = \coim \beta'.\alpha'.\]
\end{prop}
\begin{bew}
Wir prüfen die UniE des Bildes (Proposition \ref{unie-im}).
Nach UniE von $\alpha$ existiert ein Pfeil $e$ mit $\im \alpha.e = \alpha$.
Dadurch erhalten wir $\beta.\im \alpha.e = \beta.\alpha$.
Seien nun $a,b$ Pfeile, $b$ ein Mono, sodass $b.a = \beta . \alpha$.


Die verbleibenden Aussagen sind lediglich die dualen Aussagen des bereits Bewiesenen.
\end{bew}


\begin{korr}
Für $X,Y \in \mathcal A$ gelten
\begin{enumerate}
\item $\im \id_X = \id_X$
\item $\coim \id_Y = \id_Y$
\item $\im 0_{X,Y} = 0_{0,Y}$
\item $\coim 0_{X,Y} = 0_{X,0}$
\end{enumerate}
\end{korr}
\begin{bew}
Folgt direkt aus Proposition \ref{epi-mono-proj}.
\end{bew}

\begin{korr}
Seien $\alpha : X \epicr Y, \beta : Y \monicr Z$, $\alpha$ Epi und $\beta$ Mono, dann gilt $\im(\beta.\alpha) = \beta$. Insbesondere gelten $\im \beta = \beta$ und $\im \alpha = \id_Y$.

Dual dazu: $\coim(\beta.\alpha) = \alpha, \coim \alpha = \alpha, \coim \beta = \id_Y$.
\end{korr}
\begin{bew}
$\im(\beta.\alpha) = \im \beta = \beta.\im \id = \beta$. Für die anderen beiden Aussagen setzte jeweils $\alpha = \id_Y, X=Y$ bzw. $\beta = \id_Y, Y=Z$.

Für die duale Aussage dualisiere obigen Beweis.
\end{bew}
\begin{bem}
Umgekehrt folgt aus $\im\beta' = \beta'$ für einen Pfeil $\beta'$, dass $\beta'$ ein Mono ist, d.h.:
Es ist $\beta$ genau dann Mono, wenn $\im\beta = \beta$
\end{bem}

\begin{korr}
Monos sind Kerne und Kerne sind Monos.
\end{korr}
\begin{bew}
Ist $\beta$ ein Mono, so gilt $\beta = \im\beta = \ker(\coker\beta)$.
Jeder Kern $\ker(f)$ ist Mono, denn nach Proposition \ref{ker-is-fibre} ist $\ker f = \fib_1(f,0_{0,Y})$.
Der Pfeil $0_{0,Y}$ ist ein Mono wegen Proposition \ref{mono-kern-0} und Proposition \ref{ker-examples} und damit folgt aus Proposition \ref{fibre-mono}, dass auch $\fib_1(f,0_{0,Y})$ Mono ist.
\end{bew}

\begin{korr}[Idempotenz des Bildes/Cobildes]
Für jeden Pfeil $f$ gelten
\[ \im(\im f) = \im f \quad\text{und}\quad \coim(\coim f) = \coim f \]
\end{korr}
\begin{bew}
Wende Proposition \ref{epi-mono-proj} an.
\end{bew}

\begin{bem}[Epi-Mono-Faktorisierung]
Für jeden Pfeil $f$ einer abelschen Kategorie gibt es per Axiom einen Isomorphismus $\tau$, sodass $f = \im f.\tau.\coim f$.
Man kann also $\im$ als Projektion eines Pfeils $f$ mit Epi-Mono-Faktorisierung $\mu.\varepsilon$ auf seine Mono-Komponente und $\coim$ als Projektion auf seine Epi-Komponente verstehen, denn $\im f = \mu$ und $\coim f = \varepsilon$.

\end{bem}

\begin{prop} \label{im-join-im-elim}
Es gilt $\im\join(\im(g),h) = \im\join(g,h)$ für alle Pfeile $g:X\to Z,h : Y\to Z$.
\end{prop}
\begin{bew}
$\im\join(\im g,h) = \im(\join(\im g,h).\para(\coim g,\id_Y)) = \im\join(g,h)$
\end{bew}
