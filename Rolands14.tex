\section{Rolands Vierzehn}

Hier sind die 14 Rechenregeln, die uns der Prophet Roland L. auf Steintafeln gebracht hat. Das Studium dieser Aussagen über abelsche Kategorien hat dieses Handbuch motiviert.

Im folgenden sei $\mathcal A$ eine abelsche Kategorie und mit $X,
X_1 \overset{\iota_x^1}\monicr X,
X_2 \overset{\iota_x^2}\monicr X,
X'  \overset{\iota_x'}\monicr X,
Y,
Y_1 \overset{\iota_y^1}\monicr Y,
Y_2 \overset{\iota_y^2}\monicr Y,
Y'  \overset{\iota_y'}\monicr Y,
Z,
Z'  \overset{\iota_z'}\monicr Z$
seien Objekte und Unterobjekte dieser Kategorie gemeint.
Weiter seien $f:X\to Y, g:Y\to Z$ Pfeile.

\begin{defn}[Objektnotation]
%TODO
\end{defn}

\begin{prop}[Dualitätsprinzip]
%TODO
Siehe (Blatt 1, Warmup)
\end{prop}

\begin{lemm}[Regel (a)] $\quad$ % erzwinge Zeilenumbruch
\begin{enumerate}
\item $\im(0_{0,X} = 0_{0,X}$ (d.h. $f(0) = 0$)
\item $\im(f.\id_X) = \im(f)$ (d.h. $f(X) = X$)
\item $\fib(f,0_{0,Y})_1 = \ker(f)$ (d.h. $f^{-1}(0) = \Kern(f)$)
\item $\fib(f,\id_Y)_1 = \id_X$ (d.h. $f^{-1}(Y) = X$)
\end{enumerate}
\end{lemm}

\begin{lemm}[Regel (b)]
Es gelten $X_1 \cap X_2 = X_2 \cap X_1$
sowie $X_1 + X_2 = X_2 + X_1$ als Unterobjekte.
\end{lemm}
\begin{bew}
%TODO
verwende das Dualitätsprinzip etc.
\end{bew}

\begin{lemm}[Regel (c)]
Es gelten $X' \cap (X_1 \cap X_2) = (X' \cap X_1) \cap X_2$ 
sowie $X' + (X_1 + X_2) = (X' + X_1) + X_2$ als Unterobjekte.
\end{lemm}
\begin{bew}
%TODO
Formuliere universelle Eigenschaft für Dreifachschnitte bzw. Dreifachsummen, etc.
\end{bew}

\begin{lemm}[Regel (d)] $\quad$ % Zeilenumbruch
% TODO: Prüfe auf Richtigkeit!
\begin{enumerate}
\item $\fib(g.f, \iota_z')_1 = \fib(f, \fib(g, \iota_z')_1)_1$ (d.h. $(g.f)^{-1}(Z') = f^{-1}(g^{-1}(Z'))$)
\item $\im(g.f.\iota_x') = \im(g.\im(f.\iota_x'))$ (d.h. $(g.f)(X') = g(f(X'))$)
\end{enumerate}
\end{lemm}

\begin{lemm}[Regel (e)]
\[
\im(f.\fib(f,\iota_y')_1) = \im(f).\fib(\im(f),\iota_y')_1
\]
und
\[
\ldots %TODO
\]
In Objektnotation heißt dies: $f(f^{-1}(Y')) = Y' \cap \Bild(f)$, bzw. $f^{-1}(f(X')) = X' + \Kern(f)$.
\end{lemm}
