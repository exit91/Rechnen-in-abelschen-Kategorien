\section{Rolands Vierzehn}

Hier sind die 14 Rechenregeln, die uns der Prophet Roland L. auf Steintafeln gebracht hat. Das Studium dieser Aussagen über abelsche Kategorien hat dieses Handbuch motiviert.

Im folgenden sei $\mathcal A$ eine abelsche Kategorie und mit $X,
X_1 \overset{x_1}\monicr X,
X_2 \overset{x_2}\monicr X,
X'  \overset{x'}\monicr X,
Y,
Y_1 \overset{y_1}\monicr Y,
Y_2 \overset{y_2}\monicr Y,
Y'  \overset{y'}\monicr Y,
Z,
Z'  \overset{z'}\monicr Z$
seien Objekte und Unterobjekte dieser Kategorie gemeint.
Weiter seien $f:X\to Y, g:Y\to Z$ Pfeile.

\begin{defn}[Objektnotation]
Wir definieren
\begin{itemize}
\item $X_1 \cap X_2 := X_1\times_Z X_2$ als Unterobjekt von $X$ via $\fib(x_1, x_2)$.
\item $X_1+X_2 = \Bild(X_1 \oplus X_2 \to X)$ als Unterobjekt von $X$ via $\im(\join(x_1,x_2))$.
\item $f^{-1}(Y') = X \times_{f,Y,y'} Y'$ als Unterobjekt von $X$ via $\fib_1(f,y')$
\item $f(X') = \Bild(f.x')$ als Unterobjekt von $Y$ via $\im(f.x')$.
\end{itemize}
\end{defn}

\begin{bem}
Ohne Beweis habe ich oben angenommen, dass die vorkommenden Inklusionspfeile Monos sind, hier eine kurze Begründung: etc.pp. %TODO
\end{bem}

\begin{prop}[Reflexionsprinzip]
Unterobjekte und Quotientenobjekte stehen in enger Beziehung zueinander:
\begin{enumerate}
\item $\cofib(\coker(x_1), \coker(x_2)) = \coker(\im(\join(x_1,x_2)))$, (d.h. $X/X_1 \cap X/X_2 = X/(X_1+X_2)$ mit Schnitt in $\mathcal A^{op}$)
\item $\coim(\fork(\coker(x_1),\coker(x_2))) = \coker(\fib(x_1,x_2))$ (d.h. $X/X_1 + X/X_2 = X/(X_1\cap X_2)$ mit Summe in $\mathcal A^{op}$)
\item $\coim(\coker(y').f) = \coker(\fib_1(f,y'))$ (d.h. $f^{op}(Y/Y') = X/f^{-1}(Y')$ mit Bild in $\mathcal A^{op}$)
\item $\cofib_1(f,\coker(x')) = \coker(\im(f.x'))$ (d.h. $(f^{op})^{-1}(X/X') = Y/f(X')$ mit Urbild in $\mathcal A^{op}$)
\end{enumerate}
Siehe (Blatt 1, Warmup)
\end{prop}
\begin{bew}
%TODO
Beweise mich, ich bin nicht schwierig! (nur nervig!)
\end{bew}

\begin{lemm}[Regel (a)] $\quad$ % erzwinge Zeilenumbruch
\begin{enumerate}
\item $\im(0_{0,X}) = 0_{0,X}$ (d.h. $f(0) = 0$)
\item $\im(f.\id_X) = \im(f)$ (d.h. $f(X) = X$)
\item $\fib_1(f,0_{0,Y}) = \ker(f)$ (d.h. $f^{-1}(0) = \Kern(f)$)
\item $\fib_1(f,\id_Y) = \id_X$ (d.h. $f^{-1}(Y) = X$)
\end{enumerate}
\end{lemm}

\begin{lemm}[Regel (b)]
Es gelten $X_1 \cap X_2 = X_2 \cap X_1$
sowie $X_1 + X_2 = X_2 + X_1$ als Unterobjekte.
\end{lemm}
\begin{bew}
%TODO
verwende das Reflexionsprinzip etc.
\end{bew}

\begin{lemm}[Regel (c)]
Es gelten $X' \cap (X_1 \cap X_2) = (X' \cap X_1) \cap X_2$ 
sowie $X' + (X_1 + X_2) = (X' + X_1) + X_2$ als Unterobjekte.
\end{lemm}
\begin{bew}
%TODO
Formuliere universelle Eigenschaft für Dreifachschnitte bzw. Dreifachsummen, etc.
\end{bew}

\begin{lemm}[Regel (d)] $\quad$ % Zeilenumbruch
% TODO: Prüfe auf Richtigkeit!
\begin{enumerate}
\item $\fib_1(g.f, z') = \fib_1(f, \fib_1(g, z'))$ (d.h. $(g.f)^{-1}(Z') = f^{-1}(g^{-1}(Z'))$)
\item $\im(g.f.x') = \im(g.\im(f.x'))$ (d.h. $(g.f)(X') = g(f(X'))$)
\end{enumerate}
\end{lemm}
\begin{bew}
Die erste Gleichung ist nur ein Spezialfall der Staffelung von Faserprodukten.

Die zweite Gleichung folgt unmittelbar, wenn wir die Epi-Mono-Zerlegung $f.x' = \im(f.x').\varepsilon$ einsetzen:
\[\im(g.f.x') = \im(g.\im(f.x').\varepsilon) = \im(g.\im(f.x')) \]
\end{bew}

\begin{lemm}[Regel (e)]
\[
\im(\fib(f,y')) = \fib(\im(f),y')
\]
und
\[
\fib_1(f,\im(f.x')) = \im(\join(x', \ker(f)))
\]
In Objektnotation heißt dies: $f(f^{-1}(Y')) = Y' \cap \Bild(f)$, bzw. $f^{-1}(f(X')) = X' + \Kern(f)$.
\end{lemm}
\begin{bew}
Zur ersten Gleichung:
Sei eine Epi-Mono-Zerlegung $f=\im(f).\varepsilon$ gegeben.
Dann steht auf der linken Seite $\im(f.\fib_1(f,y')) = \im(\im(f).\varepsilon.\fib_1(f,y')) = \im(f).\im(\varepsilon.\fib_1(f,y'))$ und auf der rechten Seite $\im(f).\fib_1(\im(f),y')$. Es reicht also, zu zeigen, dass
\[ \im(\varepsilon.\fib_1(f,y')) = \fib_1(\im(f),y'). \]
Dazu schreiben wir erst $\varepsilon.\fib_1(f,y')$ um:
\begin{align*}
&\varepsilon.\fib_1(f,y')
\\=& \varepsilon.\fib_1(\im(f).\varepsilon,y')
\\=& \varepsilon.\fib_1(\varepsilon,\fib_1(\im(f),y'))
\\=& \fib_1(\im(f),y').\fib_2(\varepsilon,\fib_1(\im(f),y'))
\end{align*}
Der letzte Term ist eine Komposition eines Monos mit einem Epi, d.h. nach Anwendung von $\im$ überlebt nur der Mono:
\[ \im(\varepsilon.\fib_1(f,y')) = \fib_1(\im(f),y') \]
was zu zeigen war.

% Nun zur zweiten Gleichung:
% Sei $f.x' = \im(f.x').\varepsilon$ eine Epi-Mono-Faktorisierung.
% Dann faktorisiert $f.x' : X' \to Y$ über einen Pfeil $\theta : X' \to f^{-1}(f(X'))$ durch $\fib_1(f,\im(f.x'))$.
% Ebenso gilt $f.\ker(f) = \im(f.x').0$, d.h. $f.\ker(f) : \Kern(f) \to Y$ faktorisiert über einen Pfeil $\tau : \Kern(f) \to f^{-1}(f(X'))$ durch $\fib_1(f,\im(f.x'))$.
% Damit haben wir
% \begin{align*}
% \im(\join(x',\ker(f)))
% =& \im(\join(\fib_1(f,\im(f.x')).\theta,\fib_1(f,\im(f.x')).\tau))
% \\=& \im(\fib_1(f,\im(f.x')).\join(\theta,\tau))
% \\=& \fib_1(f,\im(f.x')).\im(\join(\theta,\tau))
% \end{align*}
% 
% % TODO: fix prooof
% Die Pfeile $\theta$ und $\tau$ induzieren einen Pfeil $\gamma : f^{-1}(f(X')) \to X'\oplus \Kern(f)$, $\gamma.x'$ und $\gamma.\ker(f)$ die Inklusionen der Komponenten $X', \Kern(f)$ in die direkte Summe sind.
% Nun prüfe die universelle Eigenschaft: Sei %TODO MÜÜÜÜDE
% MÜÜDEEE
% 
% Da $\fib_1(f,\im(f.x')) = \join(x',\ker(f)).\gamma$ gilt und der linke Pfeil in Mono ist, ist auch $\gamma$ ein Mono.
% Nun gilt $\im(\coprod(x',\ker(f))) = \im(\join(x',\ker(f)).\gamma) = \im(\fib_1(f,\im(f.x'))) = \fib_1(f,\im(f.x'))$, wie gewünscht.
% 
(Übung: Zeichne jeweils ein passendes kommutatives Diagramm. Das zweite könnte etwas wild werden.)
\end{bew}

\begin{lemm}[Regel (f)] $\quad$ %Zeilenumbruch
\begin{enumerate}
\item $\im(f.\im(\join(x_1,x_2))) = \im(\join(\im(f.x_1),\im(f.x_2)))$
\item $\fib_1(f,\fib(y_1,y_2)) = \fib(\fib_1(f,y_1),\fib_1(f,y_2))$
\end{enumerate}
Das bedeutet gerade: $f(X_1 + X_2) = f(X_1) + f(X_2)$ und $f^{-1}(X_1\cap X_2) = f^{-1}(X_1) \cap f^{-1}(X_2)$.
\end{lemm}
\begin{bew}
Zur ersten Gleichung: Seien $f.x_i = \im(f.x_i).\varepsilon_i$ ($i=1,2$) Epi-Mono-Zerlegungen.
Dann haben wir
\begin{align*}
   & \im(f.\im(\join(x_1,x_2)))
\\=& \im(f.\join(x_1,x_2)) \qquad \text{(nach (d))}
\\=& \im(\join(f.x_1,f.x_2))
\\=& \im(\join(\im(f.x_1).\varepsilon_1,\im(f.x_2).\varepsilon_2))
\\=& \im(\join(\im(f.x_1),\im(f.x_2)).\para(\varepsilon_1,\varepsilon_2))
\\=& \im(\join(\im(f.x_1),\im(f.x_2))) \quad \text{(eliminiere den Epi)}
\end{align*}
\end{bew}

\begin{lemm}[Regel (g), Projektionsformel] $\quad$ %Zeilenumbruch
\begin{enumerate}
\item $\fib_1(f,\im(\join(\im(f.x'),y'))) = \im(\join(x',\fib_1(f,y')))$
\item $\im(f.\fib(\fib_1(f,y'),x')) = \fib(y',\im(f.x'))$
\end{enumerate}
\end{lemm}
