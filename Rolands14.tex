\section{Rolands Vierzehn}

Hier sind die 14 Rechenregeln, die uns der Prophet Roland L. auf Steintafeln gebracht hat. Das Studium dieser Aussagen über abelsche Kategorien hat dieses Handbuch motiviert.

Im folgenden sei $\mathcal A$ eine abelsche Kategorie und mit $X,
X_1 \overset{x_1}\monicr X,
X_2 \overset{x_2}\monicr X,
X'  \overset{x'}\monicr X,
Y,
Y_1 \overset{y_1}\monicr Y,
Y_2 \overset{y_2}\monicr Y,
Y'  \overset{y'}\monicr Y,
Z,
Z'  \overset{z'}\monicr Z$
seien Objekte und Unterobjekte dieser Kategorie gemeint.
Weiter seien $f:X\to Y, g:Y\to Z$ Pfeile.

\begin{defn}[Objektnotation]
Wir definieren
\begin{itemize}
\item $X_1 \cap X_2 := X_1\times_Z X_2$ als Unterobjekt von $X$ via $x_1.\fib(x_1, x_2)_1$.
\item $X_1+X_2 = \Bild(X_1 \oplus X_2 \to X)$ als Unterobjekt von $X$ via $\im(\coprodop(x_1,x_2))$.
\item $f^{-1}(Y') = X \times_{f,Y,y'} Y'$ als Unterobjekt von $X$ via $\fib(f,y')_1$
\item $f(X') = \Bild(f.x')$ als Unterobjekt von $Y$ via $\im(f.x')$.
\end{itemize}
\end{defn}

\begin{bem}
Ohne Beweis habe ich oben angenommen, dass die vorkommenden Inklusionspfeile Monos sind, hier eine kurze Begründung: etc.pp. %TODO
\end{bem}

\begin{prop}[Reflexionsprinzip]
Unterobjekte und Quotientenobjekte stehen in enger Beziehung zueinander:
\begin{enumerate}
\item $\cofib(\coker(x_1), \coker(x_2)) = \coker(\coprodop(x_1,x_2))$, (d.h. $X/X_1 \cap X/X_2 = X/(X_1+X_2)$ mit Schnitt in $\mathcal A^{op}$)
\item $\coim(\prodop(\coker(x_1),\coker(x_2))) = \coker(\fib(x_1,x_2))$ (d.h. $X/X_1 + X/X_2 = X/(X_1\cap X_2)$ mit Summe in $\mathcal A^{op}$)
\item $\coim(\coker(y').f) = \coker(\fib(f,y'))$ (d.h. $f^{op}(Y/Y') = X/f^{-1}(Y')$ mit Bild in $\mathcal A^{op}$)
\item $\cofib(f,\coker(x')) = \coker(\im(f.x'))$ (d.h. $(f^{op})^{-1}(X/X') = Y/f(X')$ mit Urbild in $\mathcal A^{op}$)
\end{enumerate}
Siehe (Blatt 1, Warmup)
\end{prop}
\begin{bew}
%TODO
Beweise mich, ich bin nicht schwierig! (nur nervig!)
\end{bew}

\begin{lemm}[Regel (a)] $\quad$ % erzwinge Zeilenumbruch
\begin{enumerate}
\item $\im(0_{0,X}) = 0_{0,X}$ (d.h. $f(0) = 0$)
\item $\im(f.\id_X) = \im(f)$ (d.h. $f(X) = X$)
\item $\fib(f,0_{0,Y})_1 = \ker(f)$ (d.h. $f^{-1}(0) = \Kern(f)$)
\item $\fib(f,\id_Y)_1 = \id_X$ (d.h. $f^{-1}(Y) = X$)
\end{enumerate}
\end{lemm}

\begin{lemm}[Regel (b)]
Es gelten $X_1 \cap X_2 = X_2 \cap X_1$
sowie $X_1 + X_2 = X_2 + X_1$ als Unterobjekte.
\end{lemm}
\begin{bew}
%TODO
verwende das Reflexionsprinzip etc.
\end{bew}

\begin{lemm}[Regel (c)]
Es gelten $X' \cap (X_1 \cap X_2) = (X' \cap X_1) \cap X_2$ 
sowie $X' + (X_1 + X_2) = (X' + X_1) + X_2$ als Unterobjekte.
\end{lemm}
\begin{bew}
%TODO
Formuliere universelle Eigenschaft für Dreifachschnitte bzw. Dreifachsummen, etc.
\end{bew}

\begin{lemm}[Regel (d)] $\quad$ % Zeilenumbruch
% TODO: Prüfe auf Richtigkeit!
\begin{enumerate}
\item $\fib(g.f, z')_1 = \fib(f, \fib(g, z')_1)_1$ (d.h. $(g.f)^{-1}(Z') = f^{-1}(g^{-1}(Z'))$)
\item $\im(g.f.x') = \im(g.\im(f.x'))$ (d.h. $(g.f)(X') = g(f(X'))$)
\end{enumerate}
\end{lemm}

\begin{lemm}[Regel (e)]
\[
\im(f.\fib(f,y')_1) = \im(f).\fib(\im(f),y')_1
\]
und
\[
\fib(f,\im(f.x')) = \im(\coprodop(x', \ker(f)))
\]
In Objektnotation heißt dies: $f(f^{-1}(Y')) = Y' \cap \Bild(f)$, bzw. $f^{-1}(f(X')) = X' + \Kern(f)$.
\end{lemm}
